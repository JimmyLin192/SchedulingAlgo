%% This is file `elsarticle-template-1-num.tex',
%%
%% Copyright 2009 Elsevier Ltd
%%
%% This file is part of the 'Elsarticle Bundle'.
%% ---------------------------------------------
%%
%% It may be distributed under the conditions of the LaTeX Project Public
%% License, either version 1.2 of this license or (at your option) any
%% later version.  The latest version of this license is in
%%    http://www.latex-project.org/lppl.txt
%% and version 1.2 or later is part of all distributions of LaTeX
%% version 1999/12/01 or later.
%%
%% The list of all files belonging to the 'Elsarticle Bundle' is
%% given in the file `manifest.txt'.
%%
%% Template article for Elsevier's document class `elsarticle'
%% with numbered style bibliographic references
%%
%% $Id: elsarticle-template-1-num.tex 149 2009-10-08 05:01:15Z rishi $
%% $URL: http://lenova.river-valley.com/svn/elsbst/trunk/elsarticle-template-1-num.tex $
%%
\documentclass[preprint,12pt]{elsarticle}

%% Use the option review to obtain double line spacing
%% \documentclass[preprint,review,12pt]{elsarticle}

%% Use the options 1p,twocolumn; 3p; 3p,twocolumn; 5p; or 5p,twocolumn
%% for a journal layout:
%% \documentclass[final,1p,times]{elsarticle}
%% \documentclass[final,1p,times,twocolumn]{elsarticle}
%% \documentclass[final,3p,times]{elsarticle}
%% \documentclass[final,3p,times,twocolumn]{elsarticle}
%% \documentclass[final,5p,times]{elsarticle}
%% \documentclass[final,5p,times,twocolumn]{elsarticle}

%% if you use PostScript figures in your article
%% use the graphics package for simple commands
%% \usepackage{graphics}
%% or use the graphicx package for more complicated commands
%% \usepackage{graphicx}
%% or use the epsfig package if you prefer to use the old commands
%% \usepackage{epsfig}

%% The amssymb package provides various useful mathematical symbols
\usepackage{amssymb}
%% The amsthm package provides extended theorem environments
%% \usepackage{amsthm}

%% The lineno packages adds line numbers. Start line numbering with
%% \begin{linenumbers}, end it with \end{linenumbers}. Or switch it on
%% for the whole article with \linenumbers after \end{frontmatter}.
\usepackage{lineno}

%% natbib.sty is loaded by default. However, natbib options can be
%% provided with \biboptions{...} command. Following options are
%% valid:

%%   round  -  round parentheses are used (default)
%%   square -  square brackets are used   [option]
%%   curly  -  curly braces are used      {option}
%%   angle  -  angle brackets are used    <option>
%%   semicolon  -  multiple citations separated by semi-colon
%%   colon  - same as semicolon, an earlier confusion
%%   comma  -  separated by comma
%%   numbers-  selects numerical citations
%%   super  -  numerical citations as superscripts
%%   sort   -  sorts multiple citations according to order in ref. list
%%   sort&compress   -  like sort, but also compresses numerical citations
%%   compress - compresses without sorting
%%
% \biboptions{comma,round}

% \biboptions{}


\journal{CS386C Dependable Computing System}

\begin{document}

\begin{frontmatter}

%% Title, authors and addresses

%% use the tnoteref command within \title for footnotes;
%% use the tnotetext command for the associated footnote;
%% use the fnref command within \author or \address for footnotes;
%% use the fntext command for the associated footnote;
%% use the corref command within \author for corresponding author footnotes;
%% use the cortext command for the associated footnote;
%% use the ead command for the email address,
%% and the form \ead[url] for the home page:
%%
%% \title{Title\tnoteref{label1}}
%% \tnotetext[label1]{}
%% \author{Name\corref{cor1}\fnref{label2}}
%% \ead{email address}
%% \ead[url]{home page}
%% \fntext[label2]{}
%% \cortext[cor1]{}
%% \address{Address\fnref{label3}}
%% \fntext[label3]{}

\title{A Brief Survey of Multi-Processor Scheduling For Hard Real-Time Systems}

%% use optional labels to link authors explicitly to addresses:
%% \author[label1,label2]{<author name>}
%% \address[label1]{<address>}
%% \address[label2]{<address>}

\author[UTCS]{Xin Lin}
\author[UTCS]{Xiaorong Zhu}
\author[UTCS]{Lijia Liu}

\address[UTCS]{Department of Computer Science, The University of Texas at Austin}

\begin{abstract}
%% Text of abstract
In class, both of scheduling algorithms \cite{liu1973scheduling} and priority
inheritance protocols \cite{sha1990priority} in
the context of a single processor were examined in details.  Nevertheless, the
emergence and popularity of distributed computing system gave rise to the
need to solve multi-processor scheduling and priority inheritance problems. 
As the supplementary study, this paper surveys existing scheduling algorithms in
the context of multiple processors. The very first section outlines the
background of multi-processor scheduling problems, as well as system models,
terminology, and the metrics of scheduling algorithms. After that, partitioned
scheduling and global scheduling, as the primary objects of our research, will
be fully explored. Moreover, we will also give brief sketch to the hybrid
approaches of partitioned scheduling and global scheduling. 
\end{abstract}

\begin{keyword}
System \sep Scheduling Algorithm \sep Task Management
%% keywords here, in the form: keyword \sep keyword

%% MSC codes here, in the form: \MSC code \sep code
%% or \MSC[2008] code \sep code (2000 is the default)

\end{keyword}

\end{frontmatter}

%%
%% Start line numbering here if you want
%%
\linenumbers

%% main text
%%%%%%%%%%%%%%%%%%%%%%%%%%%%%%%%%%%%%%%%%%%%%%%%%%%%%%%%%%%%%%%
\section{Introduction} \label{S:1}



\subsection{Problem Defintion}

\subsection{Preview Of Related works}

\subsection{Paper Organization}
0. Background and introduction

1. System Models

2. Partitioned Scheduling

3. Global Scheduling

4. Hybrid Approach

5. Conclusion and Discussion

%%%%%%%%%%%%%%%%%%%%%%%%%%%%%%%%%%%%%%%%%%%%%%%%%%%%%%%%%%%%%%%
\section{System Models} \label{S:2}
\subsection{}

%%%%%%%%%%%%%%%%%%%%%%%%%%%%%%%%%%%%%%%%%%%%%%%%%%%%%%%%%%%%%%%
\section{Partitioned Scheduling} \label{S:3}
In this section, we will review some partitioned approaches to multiprocessor real-time scheduling.
\subsection{Characteristic of Partitioned Scheduling}
\subsection{RMNF}
\subsection{RMFF}
\subsection{EDF-FF}
\subsection{EDF-BF}
\subsection{Comparision}

%%%%%%%%%%%%%%%%%%%%%%%%%%%%%%%%%%%%%%%%%%%%%%%%%%%%%%%%%%%%%%%
\newpage
\section{Global Scheduling} \label{S:3}
In this section, we will review some global approaches to multiprocessor
real-time scheduling.

\subsection{Overview}
% what is global scheduling 

% advantages (comparison to partitioned scheduling)
The global scheduling paradigm has advantages over the partitioned approach. 
First of all, if tasks can join and leave the system at run-time, then it may be
necessary to reallocate tasks to processors in the partitioned approach.
In addition, the partitioned approach cannot produce optimal
real-time schedules -- one that meets all task deadlines
when task utilization demand does not exceed the total processor
capacity -- for periodic task sets, since the partitioning
problem is analogous to the bin-packing problem which is known to be NP-hard
in the strong sense. 
On top of that, in some embedded processor architectures with no cache and
simpler structures, the overhead of migration has a lower
impact on the performance. 
Finally, global scheduling can theoretically contribute to an increased
understanding of the properties and behaviors of real-time scheduling
algorithms for multiprocessors.

% disadvantages
The global scheduling paradigm has also several disadvantages, compared with
the partitioned approach. 
Firstly, global scheduling strategies are much more complicated to implement than
partitioned scheduling. 
In other words, for the partitioned approach, once a set of tasks are allocated
to processors, the multiprocessor real-time scheduling problem becomes a
collection of single processor real-time scheduling problems.  The ease of
programming partitioned scheduling is obvious since the single processor
scheduling problem has already been well-studied and optimal algorithms with
easy implementations already exist.
Secondly, migrating tasks at run-time means more runtime overhead in that
migrating tasks may suffer cache misses on the newly assigned processor. If
the task set is fixed and known in advanced, it is obvious that the partitioned
approach provides more appropriate solutions.

% catogries of global scheduling algorithms

% focus on global dynamic priority scheduling
Although there are various categories of global scheduling algorithms, the
focus of this paper is on the Global Dynamic Priority Scheduling. In the
following subsection, it will be chacterized in details. 

\subsection{Global Dynamic Priority Scheduling}
In this subsection, we will present our in-depth exploration to the track of
global dynamic priority scheduling algorithm. To the best of our knowledge, a
number of global dynamic priority scheduling algorithms are optimal for
periodic tasksets with explicit or implicit deadlines.
For example, Proportionate Fairness algorithm and its variants including PD,
PD$^2$, ERFair, BF, SA \cite{khemka1997optimal}, and LLREF
\cite{cho2006optimal} as well, are all optimal for offline environment.
Nevertheless, no algorithms until now are optimal to cope with online
preemptive scheduling problem, where tasksets are sporadic and multi-processor
environments are enforced.
On the other hand, despite of its optimality and dominance in theory, the
usage of global dynamic priority algorithms are limited in practice. This is
because the existence of frequent preemption and migration between tasks gives
rise to excessive overheads in potential. 

The following part of this subsection will provide brief summary of three
classic global dynamic priority scheduling algorithms. They are respectively 
Proportionate Fairness Algorithm (PFair), and Largest Local Remaining Execution
First (LLREF). 

\subsubsection{PFair}
Baruah et al \cite{baruah1996proportionate} introduced Proportionate Fairness
Algorithm. 
The Pfair class of algorithms that allow full migration
and fully dynamic priorities have been shown to be theoretically
optimal -- i.e., they achieve a schedulable utilization
bound (below which all tasks meet their deadlines) that
equals the total capacity of all processors.

\subsubsection{LLREF}
LLREF was firstly introduced by Cho et al \cite{cho2006optimal}. LLREF was
designed based on  a novel abstraction for reasoning over task execution
behavior on multiprocessors, called Time and Local Execution Time Domain
Plane (T-L Plane).

\subsection{Summary}

%%%%%%%%%%%%%%%%%%%%%%%%%%%%%%%%%%%%%%%%%%%%%%%%%%%%%%%%%%%%%%%

\section{Hybrid Approaches} \label{S:4}

\section{Conclusions} \label{S:5}

%% The Appendices part is started with the command \appendix;
%% appendix sections are then done as normal sections
%% \appendix

%% \section{}
%% \label{}

%% References
%%
%% Following citation commands can be used in the body text:
%% Usage of \cite is as follows:
%%   \cite{key}          ==>>  [#]
%%   \cite[chap. 2]{key} ==>>  [#, chap. 2]
%%   \citet{key}         ==>>  Author [#]

%% References with bibTeX database:

\newpage
\bibliographystyle{model1-num-names}
\bibliography{main}

%% Authors are advised to submit their bibtex database files. They are
%% requested to list a bibtex style file in the manuscript if they do
%% not want to use model1-num-names.bst.

%% References without bibTeX database:

% \begin{thebibliography}{00}

%% \bibitem must have the following form:
%%   \bibitem{key}...
%%

% \bibitem{}

% \end{thebibliography}


\end{document}

%%
%\begin{table}[h]
%\centering
%\begin{tabular}{l l l}
%\hline
%\textbf{Treatments} & \textbf{Response 1} & \textbf{Response 2}\\
%\hline
%Treatment 1 & 0.0003262 & 0.562 \\
%Treatment 2 & 0.0015681 & 0.910 \\
%Treatment 3 & 0.0009271 & 0.296 \\
%\hline
%\end{tabular}
%\caption{Table caption}
%\end{table}

%\begin{figure}[h]
%\centering\includegraphics[width=0.4\linewidth]{placeholder}
%\caption{Figure caption}
%\end{figure}

%\begin{equation}
%\label{eq:emc}
%e = mc^2
%\end{equation}

%j\begin{itemize}
%j\item Bullet point one
%j\item Bullet point two
%j\end{itemize}
%j
%j\begin{enumerate}
%j\item Numbered list item one
%j\item Numbered list item two
%j\end{enumerate}
